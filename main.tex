%%%%%%%%%%%%%%%%%%%%%%%%%%%%%%%%%%%%%%%%%%%%%%%%%%%%
% Template para apresentação
% Versão 2.0 (26/01/2024)
%%%%%%%%%%%%%%%%%%%%%%%%%%%%%%%%%%%%%%%%%%%%%%%%%%%%

\documentclass{beamer}

\mode<presentation> {

\usetheme{Madrid}

}

\usepackage{graphicx}
\usepackage{booktabs} % Uso de \toprule, \midrule e \bottomrule em tabelas.

%------------------------------------------------
% Primeiro Slide
\title[Subtitulo aqui]{Titulo aqui}

\author{Nome aqui}
\institute[Sigla opcional]
{
Nome da instituição aqui \\
\medskip
\textit{email aqui}
}
\date{}

\logo{\includegraphics[height=0.8cm]{Logo_UFCAT.png}}

\begin{document}

\begin{frame}
\titlepage
\end{frame}
%------------------------------------------------

%------------------------------------------------

\begin{frame}
\frametitle{Slide 1}

\begin{block}{}
Bloco de texto, equação ou lista 1
\end{block}

\begin{block}{}
Bloco de texto, equação ou lista 2
\end{block}

\end{frame}
%------------------------------------------------

\begin{frame}
\frametitle{Slide 2}

\begin{itemize}

\item Lista de itens 1

\begin{itemize}
    \item Item 1
    \item Item 2
    \item Item 3
\end{itemize}

\item Lista de itens 2

\begin{itemize}
    \item Item 1
\end{itemize}

\item Item 1
\item Item 2
\item Item 3

\end{itemize}

\end{frame}
%------------------------------------------------

\begin{frame}
\frametitle{Slide 3}

\begin{equation}
	Equacao 1
	\label{citacao1}
\end{equation}

\begin{equation}
    Equacao 2
    \label{citacao2}
\end{equation}

\end{frame}
%------------------------------------------------

\begin{frame}
\frametitle{Slide 4}

\begin{block}{Bloco de texto com título 1}
Bloco de texto, equação ou lista 1.
\end{block}

\begin{block}{Bloco de texto com título 2}
Bloco de texto, equação ou lista 2.
\end{block}

\end{frame}
%------------------------------------------------

\begin{frame}
\frametitle{Slide 5}

Inserção de imagem.
\begin{figure}[!h]
	\centering
    \caption{Título aqui}
	\includegraphics[width=0.4\linewidth]{Logo_UFCAT.png}
    \label{Citação aqui}
\end{figure}

\end{frame}
%------------------------------------------------

\begin{frame}
\frametitle{Slide 6}

\begin{block}{Bloco com título contendo exemplo de sistema}
$
y=\left\{\begin{array}{ccc}
	+ \gamma & se & x > 0\\
	- \gamma & se & x \leq 0.
\end{array}\right.
$
\end{block}
\end{frame}
%------------------------------------------------

\begin{frame}
\frametitle{Slide 7}

\begin{table}[!h]
\center
\caption{Exemplo de tabela.}
\begin{tabular}{|c|c|}
\hline 
Item 1 & Descrição ou valor do item 1. \\ 
\hline 
Item 2 & Descrição ou valor do item 2. \\ 
\hline 
\end{tabular} 
\end{table}

\end{frame}
%------------------------------------------------

\begin{frame}
\frametitle{Slide 8}

Inserção de uma imagem do lado da outra.
\begin{figure}[!h]
    \centering
    \begin{minipage}{0.5\textwidth}
        \centering
        \includegraphics[width=0.7\linewidth]{Logo_UFCAT.png} 
    \end{minipage}\hfill
    \begin{minipage}{0.5\textwidth}
        \centering
        \includegraphics[width=0.7\linewidth]{Logo_UFCAT.png}
    \end{minipage}
\end{figure}

\end{frame}
%------------------------------------------------

\begin{frame}
\frametitle{Slide 9}

Inserção de várias imagens.
\begin{figure}[!h]
    \centering
    \includegraphics[width=0.3\linewidth]{Logo_UFCAT.png}
    \includegraphics[width=0.3\linewidth]{Logo_UFCAT.png}
    \includegraphics[width=0.3\linewidth]{Logo_UFCAT.png}
    \includegraphics[width=0.3\linewidth]{Logo_UFCAT.png}
    \includegraphics[width=0.3\linewidth]{Logo_UFCAT.png}
\end{figure}

\end{frame}
%----------------------------------------------

\end{document}
